

% \title{Documentation technique -- Complétion automatique}
% \author{}
% \date{}

% \begin{document}

% \maketitle

\section*{Description générale}

Ce projet implémente un outil de complétion semi-automatique de texte, utilisable dans toutes les applications sur un système Ubuntu.

\section*{Installation}

\subsection*{Prérequis}

\begin{itemize}
	\item Système \textbf{Ubuntu} (testé sur Ubuntu 22.04.2 LTS)
	\item Le raccourci clavier \textbf{Alt + Tab} doit être actif (natif sur Ubuntu) pour changer d'application facilement
\end{itemize}

\subsection*{Étapes d'installation}

\begin{lstlisting}[language=bash]
cd completion-system
make install
\end{lstlisting}

Le \texttt{Makefile} s’occupe de :
\begin{itemize}
	\item Installer les dépendances nécessaires
	\item Copier le binaire compilé dans \texttt{/usr/local/bin/} (inclus dans le PATH)
	\item Donner les bons droits d’exécution à tous les fichiers nécessaires
	\item Créer le fichier .desktop pour le menu d’applications
\end{itemize}

\section*{Lancement}

Deux méthodes sont possibles :

\begin{itemize}
	\item \textbf{Depuis le terminal} :
	      \begin{lstlisting}[language=bash]
completion-system
  \end{lstlisting}

	\item \textbf{Depuis le gestionnaire d'applications Ubuntu} :
	      Ouvrir le menu et rechercher \texttt{Completion System}, puis appuyer sur Entrée.
\end{itemize}


\textbf{Remarque :} sur certains ordinateurs, il est possible que le lancement via le gestionnaire d’applications ne fonctionne pas. Dans ce cas, le terminal reste pleinement fonctionnel.

Une fois lancé, le programme tourne en arrière-plan et fonctionne dans \textbf{toutes les applications}.

\section*{Désinstallation}

\begin{lstlisting}[language=bash]
cd completion-system
make uninstall
\end{lstlisting}

\section*{Binaire autonome}

Le fichier binaire \texttt{completion-system}, déjà compilé et fourni dans le dossier \texttt{bin/}, est entièrement autonome après installation.

\begin{itemize}
	\item Il est copié dans \texttt{/usr/local/bin}, ce qui permet de le lancer depuis n'importe où.
	\item Il ne dépend plus des fichiers sources ni du scripts Python.
\end{itemize}

\section*{Structure des fichiers}

L'arborescence du projet est la suivante :

\begin{lstlisting}[basicstyle=\ttfamily]
.
|- data/
|       |- dico\_freq.csv
|- src/
|       |- gui.py
|       |- keylogger.rs
|       |- main.rs
|       |- mouselogger.rs
|       |- offset.rs
|       |- python_gui.rs
|       |- suggestions.rs
|       |- virtual_input.rs
|- tools/
|       |- format_dico.py
|       |- Lexique383.tsv
|- Cargo.toml
|- completion-system.png
|- Makefile
|- README.md
\end{lstlisting}


\begin{itemize}
	\item \texttt{src/} : contient tous les fichiers source Rust ainsi que le script Python pour l'interface graphique (\texttt{gui.py}).
	\item \texttt{data/dico\_freq.csv} : dictionnaire final utilisé pour la complétion.
	\item \texttt{tools/Lexique383.tsv} : base de données brute téléchargée.
	\item \texttt{tools/format\_dico.py} : script Python utilisé pour convertir le fichier \texttt{.tsv} en \texttt{.csv}, il permet de plus d'enlever les informations inutiles.
	\item \texttt{Cargo.toml} : fichier de configuration qui indique les dépendances Rust et leurs versions.
	\item \texttt{completion-system.png} : icône affichée dans le menu d’applications Ubuntu.
\end{itemize}

% \end{document}

