\documentclass[a4paper, 11pt]{report}
\usepackage[T1]{fontenc} % Caractere francais
\usepackage[utf8]{inputenc}
\usepackage[english,french]{babel} 
\usepackage{graphicx} % Pour les images
\usepackage{multicol} % Pour faire des multi colonnes
\usepackage[export]{adjustbox} % Pour la clé 'valign'  (aligner verticalement)
\usepackage[colorlinks=true,linkcolor=black]{hyperref} % Pour qu'il y est des liens sur la table des matières
\usepackage{caption} % Utiliser plus de fonction sur caption (caption* pour ne pas afficher FIGURE-1)
\usepackage{lipsum} % Pour générer du texte pour voir comment ca rend
\usepackage{ragged2e} % Pour justifier le texte
\usepackage{ulem}
\usepackage[margin=3.2cm]{geometry}

\usepackage{listings}
\usepackage{xcolor}





\begin{document}


% Première page
\begin{titlepage}
    \centering
    \vspace*{2cm}
    {\Huge \textbf{Université de Franche-Comté} \par}
    \vspace{1cm}
    {\huge \texttt{}{Projet d'Initiation à la recherche\\ } \LARGE{\textbf{L2 - CMI}} \par} 
    \vspace{1.2cm}
    {\huge \textbf{Complétion (semi-)automatique} \par}
    \vspace{1.5cm}
    {\Large BOITEUX Elouan\\BENALI Samia\par}
    \vspace{5cm}
    \begin{minipage}[c]{0.40\textwidth}
        \centering
        \raisebox{-0.5\height}{\includegraphics[width=\textwidth]{latex-images/logo_univ.png}}
    \end{minipage}
    \hfill
    \begin{minipage}[c]{0.5\textwidth}
        \centering
        \raisebox{-0.5\height}{\includegraphics[width=\textwidth]{latex-images/logo_CMI.png}}
    \end{minipage}
    \vfill
    {2024/2025}
\end{titlepage}

\tableofcontents



\chapter*{Introduction} % * pour ne pas avoir de numéro de chapitre
\addcontentsline{toc}{chapter}{Introduction}

Défis dans la complétion semi-automatique : 
\begin{itemize}
    \item Assurer des suggestions précises et en adéquation avec l'intention e l'utilisateur
    \item Trouver le juste milieu entre précision et rapidité
    \item Gestion des langues et des styles
\end{itemize}


\chapter{Les différentes approches utilisées jusqu'à aujourd'hui}
\section{Modèles basés sur des règles}
Pour proposer des suggestions, ce modèle utilisent des algorithmes simples basé sur la correspondance de préfixes ou l'application de règles syntaxiques présices permettant de générer des complétions basées sur des motifs connus.

\section{Modèles statistiques}
a reformuler
Les modèles statistiques exploitent les données historiques pour estimer les complétions probables. Les N-grammes ou les chaînes de Markov sont couramment utilisés pour capturer les dépendances locales dans les séquences.

\section{Modèles d’apprentissage automatique}
a reformuler
Avec l’apprentissage supervisé et non supervisé, ces modèles apprennent des motifs complexes à partir des données. Ils peuvent inclure des algorithmes comme les forêts aléatoires ou les régressions pour fournir des prédictions plus contextuelles.

\section{Modèles de deep learning}
a reformuler
Les réseaux de neurones avancés, tels que les RNN (et leurs variantes LSTM et GRU), ainsi que les modèles Transformers comme GPT, offrent une capacité à comprendre le contexte global et à générer des complétions précises.


\chapter{Les algorythmes de calcul de distance}

\section{Distance d'édition}

\chapter{Chaine de Markov}

\chapter{Notre outil de complétion (semi-)automatique}

\chapter*{Conclusion}
\addcontentsline{toc}{chapter}{Conclusion} 
% Page de résumé
\newpage
\begin{center}
    \vspace*{\fill} % Espace vertical
    \section*{Résumé}
    \addcontentsline{toc}{chapter}{Résumé}
    \begin{justify}
Le resumé
    \end{justify}
\end{center}

\end{document}