\documentclass[10pt]{beamer}


\usepackage{graphicx} % Pour les images


\usetheme[progressbar=frametitle]{metropolis}
\usepackage{appendixnumberbeamer}
\usepackage{booktabs}
\usepackage[scale=2]{ccicons}
\usepackage{pgfplots}
\usepgfplotslibrary{dateplot}
\usepackage{xspace}
\usepackage{tikz} % Pour les flèches et les diagrammes
\usepackage{array} % Pour les tableaux
\usepackage{eurosym}
\usepackage{enumitem}
\usepackage{tikz}
\usepackage{fontenc}


% Redéfinir la commande \insertframenumber pour inclure le logo
\addtobeamertemplate{frametitle}{}{
  \tikz[remember picture,overlay] 
    \node[anchor=north east,inner sep=0cm] at (current page.north east) {
      \includegraphics[height=0.97cm]{images/logo.png}
    };
}
% Définir les styles par défaut
\setlist[itemize,1]{label=\textbullet} % Premier niveau : point plein
\setlist[itemize,3]{label=\textendash} % Deuxième niveau : tiret long
\setlist[itemize,2]{label=$\circ$} % Troisième niveau : cercle vide

% Définir les styles par défaut pour enumerate
\setlist[enumerate,1]{label=\arabic*.} % Premier niveau : chiffres arabes
\setlist[enumerate,2]{label=\Alph*.} % Deuxième niveau : lettres majuscules
\setlist[enumerate,3]{label=\alph*.} % Troisième niveau : lettres minuscules

% Logo à gauche et à droite
\titlegraphic{
  \includegraphics[height=1.3cm]{images/logo_univ.png}\hfill
  \includegraphics[height=1.3cm]{images/logo_CMI.png}\\[3cm] % Espace entre les logos du haut et le logo Cibest
  \vspace{0.1cm} % Ajoute un espace vertical pour descendre le logo Cibest plus bas
  \hfill\includegraphics[height=3.3cm]{images/illustration.png}
}



\title{\vspace{1cm}Complétion semi-automatique\\Projet d'Initiation à la recherche}
\author{Samia Benali \& Elouan BOITEUX\\[0.3cm]Année universitaire 2024 -- 2025 \\[0.3cm]CMI Informatique Deuxième Année}
\date{}



\begin{document}

\maketitle

\begin{frame}{Table des matières}
	\setbeamertemplate{section in toc}[sections numbered]
	\tableofcontents%[hideallsubsections]
\end{frame}


\section{L'entreprise}

\begin{frame}{Contexte de l'entreprise}
    
    \begin{center}
        \vspace*{-0.2cm}
        \includegraphics[width=0.9\textwidth]{images/cibest_group.png}
        \hspace*{-0.5cm}
        \includegraphics[width=1.1\textwidth]{images/frise_cibest.png}
    \end{center}
  
\end{frame}

\begin{frame}{Cibest}
    \begin{columns}
        \column{0.45\textwidth}
        
\textbf{Quelques chiffres~:}
\begin{itemize}
    \item 3 sites~: Belfort, Besançon, Les Ulis
    \item 1 filiale en Angleterre
    \item Plus de 70 collaborateurs
    \item Chiffre d'affaires~: 10 M\euro{}
    \item 30 ans d'expérience
\end{itemize}
\vspace{0.6cm}
\textbf{Principaux clients~:\\[0.3cm]}
\hspace*{-0.7cm}
\includegraphics[width=1.2\textwidth]{images/clients_cibest.png}

        \column{0.45\textwidth}
        \includegraphics[width=\textwidth]{images/carte_cibest.png}
        
    \end{columns}
\end{frame}

% \begin{frame}{Cibest}
%     \begin{columns}[T]
%         \metroset{block=fill}
%         \column{0.45\textwidth}
%         \begin{exampleblock}{Cibest Solution}
%                 \textbf{Champs d'action~:}
%                 \begin{itemize}
%                     \item Domaine du logiciel\\embarqué
%                     \item Vidéoprotection
%                     \item Wifi passagers
%                     \item Comptage passagers
%                     \item Rétrovision numérique
%                 \end{itemize}

%                 \textbf{Statistiques~:}
%                 \begin{itemize}
%                     \item + de 450 clients équipés
%                     \item + de 200 villes en France
%                 \end{itemize}
%           \end{exampleblock}
        
%         \column{0.45\textwidth}
%         \begin{alertblock}{Cibest IT}
%             \textbf{Champs d'action~:}
%             \begin{itemize}
%                 \item Informatique dans les\\transports
%                 \item Billetique
%             \end{itemize}

%             \textbf{Statistiques~:}
%             \begin{itemize}
%                 \item + de 30 consultants clients équipés
%                 \item + de 200 villes en France
%             \end{itemize}
%           \end{alertblock}
%     \end{columns}
%     \end{frame}



\section{Missions du Stage}
\begin{frame}{Contexte et Problématique Actuelle}
    \begin{columns}
        \column{0.5\textwidth}
            \textbf{Contexte}
            \begin{itemize}
                \item Installation d'équipements
                \item Configurations personnalisées
                \item Sauvegarde des configurations
                \item Identification des problèmes
            \end{itemize}
        
        \column{0.5\textwidth}
            \textbf{Problématique}
            \begin{itemize}
                \item Récupération manuelle
                \item Perte de temps, erreurs
                \item Résultats sous diverses formes
                \item Absence de contrôle qualité
            \end{itemize}
    \end{columns}
    \vfill
    \centering\includegraphics[width=0.9\textwidth]{images/camera_bus.png}
\end{frame}

% Diapositive 2 : Besoins et Solution Proposée
\begin{frame}{Besoins Identifiés et Solution Proposée}
    \begin{columns}
        \column{0.7\textwidth}
        \hspace*{-0.1\textwidth}
            \includegraphics[width=1.1\textwidth]{images/besoins_stage.pdf}
        \column{0.3\textwidth}
        \hspace*{-0.23\textwidth}
            \includegraphics[width=1.4\textwidth]{images/solution_stage.pdf}
    \end{columns}
\end{frame}
% Diapositive 2 : Besoins et Solution Proposée
\begin{frame}{Besoins Identifiés et Solution Proposée}
    \begin{columns}
        \column{0.3\textwidth}
        \hspace*{-0.13\textwidth}
            \includegraphics[width=1.2\textwidth]{images/besoins_stage.pdf}
            
        \column{0.8\textwidth}
        
            \hspace*{-0.75cm}
            \includegraphics[width=1.125\textwidth]{images/solution_stage.pdf}

    \end{columns}
\end{frame}


\begin{frame}{Méthodologie de Travail}
  \begin{columns}[t]
    \column{0.5\textwidth}
    \textbf{Méthode Agile~:}\\[0.4cm]
    \hspace*{-0.4cm}
    \vspace*{-1cm}
    \begin{itemize}
        \item flexibilité
        \item adaptabilité
        \item amélioration continue
    \end{itemize}
    \includegraphics[width=1.1\textwidth]{images/schema_conception.png}
    \column{0.5\textwidth}
    \textbf{Outils Utilisés~:}
    \hspace*{0.2cm}
    \includegraphics[width=0.8\textwidth]{images/logo_outils.png}
  \end{columns}
\end{frame}
\section{Les différentes approches utilisées}
\begin{frame}{Modèle s'appuyant sur des règles}
	\begin{itemize}
		\item Utilisation de règles prédéfinies
		      \begin{itemize}
			      \item préfixe
			      \item séquence en amont
		      \end{itemize}
		\item Avantages~: simplicité et rapidité de mise en œuvre
		\item Inconvénients~: rigidité, difficulté à gérer des cas complexes
	\end{itemize}
\end{frame}

\begin{frame}{Modèle s'appuyant sur des statistiques}
	\begin{itemize}
		\item Prédiction grâce aux données d'un historique
		\item Avantages~: flexibilité, capacité à gérer des données complexes
		\item Inconvénients~: nécessite un grand historique de donées
	\end{itemize}
\end{frame}


\begin{frame}{Modèle s'appuyant sur l'intelligence artificielle}
	\begin{itemize}
		\item Utilisation de l'intelligence artificielle et les réseaux de neurones
		\item Avantages~: capacité à apprendre des modèles complexes, adaptabilité
		\item Inconvénients~: énormement de temps de calcul et de ressources
	\end{itemize}
\end{frame}


\begin{frame}{Modèle s'appuyant sur le deep learning}
	\begin{itemize}
		\item
		\item Avantages~: performances élevées
		\item Inconvénients~: dépend des données collectées, implémentation complexe
	\end{itemize}
\end{frame}

\section{Les algorithmes de calcul de distance}
% Un algorithme de calcul de distance permet de mesurer la similarité et/ou la dif-
% férence entre deux objets tels que du texte, des vecteurs, des chaînes de caractères.
% On utilise ces algorithmes de calcul principalement pour la correction d’orthographe,
% le traitement de texte, les alignements de séquences ADN ou encore pour la recherche
% d’information


\begin{frame}{Distance de Levenshtein}
	\textbf{Objectif : }Trouver le nombre minimal d'opérations nécessaires pour transformer une chaîne de caractères en une autre.
	\begin{center}
		\includegraphics[width=0.9\textwidth]{images/levenshtein.png}
	\end{center}
	\textbf{Complexité : }  $\mathcal{O}(n \times m)$ (où n et m sont les longueurs des chaînes de caractères)
\end{frame}

\begin{frame}{Distance de Damerau-Levenshtein}
	\vspace*{-0.3cm}
	\textbf{Objectif : }identique à la distance de Levenshtein, mais avec l'ajout de la possibilité d'échanger deux caractères adjacents.
	\begin{center}
		\includegraphics[width=0.9\textwidth]{images/damerau-levenshtein.png}
	\end{center}
	\textbf{Complexité : }  $\mathcal{O}(n \times m)$ (où n et m sont les longueurs des chaînes de caractères)
\end{frame}

\begin{frame}{Distance de Hamming}
	\begin{columns}
		\column{0.5\textwidth}
		Compter le nombre de positions où les caractères diffèrent.
		\column{0.5\textwidth}
		Utilisable que pour des mots de même longueur.
	\end{columns}
	\begin{center}
		\includegraphics[width=0.9\textwidth]{images/hamming.png}
	\end{center}
	\textbf{Complexité : }  $\mathcal{O}(n)$ (où n est la longueur des chaînes de caractères)
\end{frame}

\section{Les chaînes de Markov}
% Un algorithme de calcul de distance permet de mesurer la similarité et/ou la dif-
% férence entre deux objets tels que du texte, des vecteurs, des chaînes de caractères.
% On utilise ces algorithmes de calcul principalement pour la correction d’orthographe,
% le traitement de texte, les alignements de séquences ADN ou encore pour la recherche
% d’information


\begin{frame}{Définition d'une chaîne de Markov}
	\begin{itemize}
		\item \textbf{Chaîne de Markov} : Modèle mathématique représentant un système de probabilité. Les probabilités de passer d'un état à un autre dépendent entièrement de l'état actuel.
		\item \textbf{Etats} : Elements du système.
		\item \textbf{Transition} : Action de passer d'un état à un autre.
		\item \textbf{Matrice transition} : Table permettant de regrouper les transitions entre tous les états.
	\end{itemize}
	\begin{center}
		\includegraphics[width=0.38\textwidth]{images/def_markov.png}
		\vspace*{0.3cm}
	\end{center}

\end{frame}

\begin{frame}{Application}
	\vspace*{-0.3cm}
	Utilisation pour modélisation de processus aléatoires, analyse de séquences, de prédictions ou d'historiques de navigation.
	Analyse des historiques de navigation ainsi que les actions de l'utilisateur.

	\begin{center}
		\includegraphics[width=0.6\textwidth]{images/tableau_proba_markov.png}
		\par
		\uline{Exemple transitions entre trois pages web A, B et C}
	\end{center}
\end{frame}

\begin{frame}{Matrice transition}
	Modélisation des transitions probables entre les étapes.
	\begin{itemize}
		\item Collecter les données
		\item Compter les transitions
		\item Calcul des probabilités
		\item Construction de la matrice
	\end{itemize}
	\begin{center}
		\includegraphics[width=0.6\textwidth]{images/matrice_transition.png}
		\par
		\uline{Résultat selon le tableau précédent}
	\end{center}
\end{frame}

\begin{frame}{Exemple complet}
	\vspace{-0.5cm}
	\centering
	\includegraphics[width=0.8\textwidth]{images/transition_ex_markov.png}
	\vspace{1cm}
	\begin{columns}
		\column{0.6\textwidth}
		\includegraphics[width=\textwidth]{images/tableau_ex_markov.png}
		\column{0.4\textwidth}
		\includegraphics[width=\textwidth]{images/matrice_markov.png}
	\end{columns}
	\vspace{1cm}
	\textbf{Calcul probabilités} :  nombre de fois où la transition \(X\) vers \(Y\) est observée, puis de diviser par le total des transitions partant de l'état \(X\).
\end{frame}

\section{Notre outil de complétion semi-automatique}

\begin{frame}{Choix pour l'implémentation}
	\begin{columns}
		\column{0.5\textwidth}
		\textbf{Notre objectif~:}
		\begin{itemize}
			\item Reproduire un outil de complétion semi-automatique comme sur téléphone mais sur ordinateur
			\item Utilisable dans n'importe quel application
		\end{itemize}

		\column{0.5\textwidth}
		\begin{center}
			\includegraphics[width=\textwidth]{images/suggestion_windows.png}
			\uline{Outil de suggestion de mots sur Windows}
		\end{center}
	\end{columns}
	\vspace{0.5cm}
	\begin{columns}
		\column{0.5\textwidth}
		\textbf{Choix du langage~:}
		\begin{itemize}
			\item Apprendre un nouveau langage
			\item Langage de programmation moderne
			\item Conçu pour la performance
		\end{itemize}

		\column{0.5\textwidth}
		\includegraphics[width=\textwidth]{images/rust.jpeg}
	\end{columns}
\end{frame}


\begin{frame}{Création du keylogger \& mouselogger}
	\vspace*{-0.5cm}
	\begin{columns}
		\column{0.5\textwidth}
		\begin{itemize}
			\item Détection des péréphériques
			\item Lecture des évenements
			\item Décodage avec le fichier \textbf{input-event-codes.h}
			\item Sauvegarde des évènements pour récupérer le mot tapé
		\end{itemize}
		\column{0.5\textwidth}
		\includegraphics[width=\textwidth]{images/fichier_touche.png}
	\end{columns}
	\begin{center}
		\vspace*{-0.3cm}
		\includegraphics[width=\textwidth]{images/lecture-fichier-borderless.png}
	\end{center}

\end{frame}



\begin{frame}{Création du clavier virtuel}
	\textbf{Objectif~:} Ecrire le mot que l'utilisateur a choisi sur l'interface de l'application
	\begin{itemize}
		\item Traduction caractère → évènement clavier
		\item Envoi séquentiel des lettres du mot
		\item Suppression du mot précédent (retour arrière $n$ fois)
		\item Réécriture fluide et invisible
	\end{itemize}
\end{frame}


\begin{frame}{Algorithme de suggestion}
	Utilisation de la \textbf{distance de Levenshtein} pour la suggestion de mots

	\begin{itemize}
		\item Comparaison avec $\simeq$ 140 000 mots
		\item Sélection des mots les plus proches
	\end{itemize}

	\begin{block}{\alert{Résultats décevants}}
		→ Suggestions peu pertinentes\\
		→ Nécessité d’ajouter des critères complémentaires
	\end{block}

	\textbf{Solutions apportées~:}
	\begin{itemize}
		\item Analyse des \textbf{préfixes}
		\item Pondération selon la \textbf{fréquence d'utilisation}
	\end{itemize}
\end{frame}

\begin{frame}{Algorithme de suggestion}
	\vspace*{-0.75cm}
	\begin{center}

		\includegraphics[height=0.98\textheight]{images/uml_diapo.png}
	\end{center}

\end{frame}




\begin{frame}{L'interface graphique}
	\begin{columns}
		\column{0.45\textwidth}
		\begin{itemize}
			\item Initialement, interface prévue en \textbf{Rust avec GTK}
			\item Problème : impossible de garder la fenêtre au premier plan constamment
			\item Solution : interface réalisée en \textbf{Python avec Tkinter}
			\item Communication entre Rust et Python via leur \textbf{entrée standard}
		\end{itemize}

		\column{0.50\textwidth}
		\centering
		\textbf{Interface finale :} \\
		\includegraphics[width=\textwidth]{images/demo_interface.png}
	\end{columns}
\end{frame}


\begin{frame}{Installeur de l'application}
	\begin{itemize}
		\item Création d'un installeur pour l'application
		\item Utilisation d'un \textbf{Makefile}
		\item Pour
		      \begin{itemize}
			      \item Compiler le code
			      \item Attribuer les droits nécessaire
			      \item Créer le fichier \textbf{.desktop}
			      \item Déplacer le binaire
		      \end{itemize}
	\end{itemize}
	\begin{center}
		\includegraphics[width=0.5\textwidth]{images/application.png}
	\end{center}

\end{frame}

\section{Démonstration}% montrer la video

\section*{Questions}
\begin{frame}{Questions~?}
	\vspace*{-0.3cm}
	\begin{columns}
		\column{0.5\textwidth}
		\begin{center}
			\includegraphics[width=0.6\textwidth]{images/logo_univ.png}
		\end{center}
		
		\column{0.5\textwidth}
		\begin{center}
			\includegraphics[width=0.6\textwidth]{images/logo_CMI.png}
		\end{center}
		
	\end{columns}
	\vspace*{0.2cm}
	
	\centering
	\emph{Merci de votre attention~!}\\
	\vspace{0.6cm}
	\textbf{Avez-vous des questions~?}
	\includegraphics[width=0.7\textwidth]{images/questions.png}
\end{frame}



\end{document}
